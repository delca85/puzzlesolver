\documentclass[a4paper]{article}
\usepackage[italian]{babel}
\usepackage[T1]{fontenc}
\usepackage[utf8]{inputenc}
\usepackage{cite}
\usepackage{hyperref}
\newcommand{\Classname}[1]{\textsc{#1}}
\newcommand{\Ifacename}[1]{\textsc{#1}}
\newcommand{\Methodname}[1]{\texttt{#1}}
\newcommand{\Pkgname}[1]{\texttt{#1}}
\usepackage{listings} % Include the listings-package
\usepackage{tikz}
\usetikzlibrary{arrows}

\lstset{
language=Java,
basicstyle=\small\sffamily,
numbers=left,
numberstyle=\tiny,
frame=tb,
columns=fullflexible,
showstringspaces=false
}

\usepackage{tikz}
\usetikzlibrary{shapes,arrows}
\tikzstyle{frame} = [
rectangle, draw,
text width=4em, text centered,
minimum height=3em
% minimum width=5em
]
\tikzstyle{line} = [draw, -latex']


\title{PuzzleSolver - Relazione Programmazione Concorrente e Distribuita}
\author{Tobia Tesan - \#1051819}
\begin{document}
\maketitle
\tableofcontents

\begin{abstract}
La presente relazione dettaglia e motiva le scelte progettuali dell'allegato programma \texttt{PuzzleSolver}, che realizza le specifiche della parte 1 del progetto di Programmazione Concorrente e Distribuita per l'A.A. 2014/2015 \cite{prspec}.
\end{abstract}

\section{Vista d'insieme}

Il progetto consta di 11 classi principali, di cui 2 interfacce, 1 classe astratta e 3 eccezioni, organizzate in due package a seconda della funzionalit\`a (I/O e logica) nel modo seguente:

\begin{verbatim}
puzzlesolver/
|-- core
|   |-- BasicPuzzlePiece.java
|   |-- BasicPuzzlePieceTest.java
|   |-- BFSHashmapPuzzle.java
|   |-- BFSHashmapPuzzleTest.java
|   |-- HashmapPuzzle.java
|   |-- IPuzzle.java
|   |-- IPuzzlePiece.java
|   |-- MissingPiecesException.java
|   `-- PuzzleNotSolvedException.java
|-- io
|   |-- IPuzzlePrinter.java
|   |-- MalformedFileException.java
|   |-- PlaintextPuzzlePrinter.java
|   `-- PuzzleFileParser.java
`-- PuzzleSolver.java
\end{verbatim}

Ad esse si aggiungono due piccole classi annidate di utilit\`a, il main e alcuni unit test.

Il perno principale del progetto sono le interfacce \Ifacename{Puzzle} e \Ifacename{IPuzzlePiece}: le classi che le implementano rappresentano rispettivamente un singolo pezzo di puzzle e una classe contenitrice.

\section{Dettaglio delle classi}
\subsection{\Pkgname{puzzlesolver.core}}
\subsubsection{\Classname{IPuzzlePiece}}
\Ifacename{IPuzzlePiece} prescrive dei metodi di ispezione dell'ID del pezzo, dell'ID dei suoi vicini e alcuni semplici helper per ottenere informazioni sintetiche come \Methodname{isNWCorner()}.

Per natura del task \`e fatta l'assunzione che gli ID del pezzo e dei vicini non possiano cambiare e dunque l'interfaccia non prescrive metodi di inserimento e modifica di tali informazioni.

L'interfaccia prescrive tuttavia dei metodi \Methodname{getNorth()} ... \Methodname{getEast()}  e  \Methodname{getNorth()} ... \Methodname{getEast()} che permettono di leggere e modificare riferimenti ai vicini man mano che il puzzle viene risolto.

Ogni \Ifacename{IPuzzlePiece} pu\`o essere visto come nodo di un grafo di grado massimo (uscente) 4 e inizialmente 0 in cui i riferimenti ai vicini sono archi.

Per esempio, lo stato finale desiderato per i quattro pezzi del puzzle $2\times2$ ``Ciao'' \`e quello in figura \ref{fig:ciao}.

\begin{figure}[h!]
  \centering
\begin{tikzpicture}[->,>=stealth',shorten >=1pt,auto,node distance=3cm,
  thick,main node/.style={circle,draw}]
  \node[main node] (1) {C};
  \node[main node] (n1)[above left of=1] {\texttt{null}};
  \node[main node] (3) [below of=1] {a};
  \node[main node] (2) [right of=1] {i};
  \node[main node] (4) [right of=3] {o};
  \node[main node] (n4)[below right of=4] {\texttt{null}};
  \path[every node/.style={font=\sffamily\small}]
    (1)  edge [bend right] node[below] {E} (2)
         edge [bend right] node[left] {S} (3)
         edge [bend right] node[left] {N} (n1)
         edge [bend left] node[left] {W} (n1)
    (2)  edge [bend right] node[above] {W} (1)
         edge [bend right] node[left] {S} (4)
         edge [bend right] node[above] {N} (n1)
         edge [bend left] node[above] {E} (n4)
    (3)  edge [bend right] node[left] {N} (1)
         edge [bend right] node[below] {E} (4)
         edge [bend right] node[above] {S} (n4)
         edge [bend left] node[above] {W} (n1)
    (4)  edge [bend right] node[left] {N} (2)
         edge [bend right] node[below] {W} (3)
         edge [bend right] node[left] {S} (n4)
         edge [bend left] node[left] {E} (n4);
\end{tikzpicture}
\caption {Un puzzle $2\times2$}
\label{fig:ciao}
\end{figure}

\begin{figure}[h]
  \centering
\begin{lstlisting}[frame=single]
public interface IPuzzlePiece {
	String getId();
	char getCharacter();
	IPuzzlePiece getNorth();
	IPuzzlePiece getSouth();
	IPuzzlePiece getWest();
	IPuzzlePiece getEast();
	void setNorth(IPuzzlePiece p);
	void setSouth(IPuzzlePiece p);
	void setWest(IPuzzlePiece p);
	void setEast(IPuzzlePiece p);
	String getNorthId();
	String getSouthId();
	String getEastId();
	String getWestId();
	boolean isNRow();
	boolean isWCol();
	boolean isSRow();
	boolean isECol();
	boolean isNWCorner();
	boolean isSWCorner();
	boolean isSECorner();
	boolean isNECorner();
}
\end{lstlisting}
\caption {\textsc{IPuzzlePiece}}
\end{figure}

\subsubsection{\Classname{IPuzzle}}
L'interfaccia \Ifacename{IPuzzle} espone i metodi \Methodname{addPiece}, \Methodname{solve}, \Methodname{getSolution}, \Methodname{Iterator}, \Methodname{getRows} e \Methodname{getCols}.

Il flusso di programma che si intende realizzare \`e:

\begin{enumerate}
\item Istanziazione
\item Aggiunta dei pezzi con \Methodname{addPiece}
\item Calcolo di una soluzione con \Methodname{solve}
\item Output di una soluzione con \Methodname{getSolution}, \Methodname{getIterator}, \Methodname{getRows}, \Methodname{getCols}.
\end{enumerate}

A questo scopo, i metodi di cui al quarto punto sollevano una \Classname{RuntimeException} chiamata \Classname{PuzzleNotSolvedException} - l'invocazione di tali metodi prima dell'avvenuto calcolo di una soluzione costituisce un \emph{errore logico} da parte del programmatore.
\label{pnse}
Il metodo \Methodname{solve} pu\`o sollevare una \Classname{MissingPiecesException} se non \`e possibile calcolare una soluzione per mancanza di pezzi.

Non \`e prevista una eccezione da sollevare in caso di pezzi \emph{in eccesso} o di errori logici pi\`u  sottili nel puzzle, anche se non \`e nemmeno esclusa (ne \`e possibile escludere) una RuntimeException.

\begin{figure}[h]
  \centering
\begin{lstlisting}[frame=single]
public interface IPuzzle {
	public void addPiece(IPuzzlePiece p);
	public void solve() throws MissingPiecesException;
	public Iterator<Iterator<PuzzlePiece>> iterator();
	public String getSolution() throws PuzzleNotSolvedException;
	public int getRows() throws PuzzleNotSolvedException;
	public int getCols() throws PuzzleNotSolvedException;
}
\end{lstlisting}
\caption {\textsc{IPuzzle}}
\end{figure}

\subsubsection{\Classname{BasicPuzzlePiece}}
\Classname{BasicPuzzlePiece} \`e una triviale implementazione dell'interfaccia \Ifacename{IPuzzlePiece} costruita intorno a due array di 4 elementi.
Si rimanda al codice per i dettagli.

\subsubsection{\Classname{HashmapPuzzle}}
\Classname{HashmapPuzzle} \`e una classe astratta che implementa \Ifacename{IPuzzle}.

Fornisce una Hashmap dove salvare dei \Ifacename{IPuzzlePiece}  e recuperarli a costo $\approx O(1)$ nel caso medio e una serie di metodi di utilit\`a.

Ha una classe annidata privata \Classname{BasicPuzzleIterator} di utilit\`a che implementa l'interfaccia \Ifacename{Iterator<Iterator<PuzzlePiece>>}. Non \`e stato ritenuto necessario renderla standalone visto che \`e fortemente dipendente da \Classname{BasicPuzzle}, n\`e \`e stato ritenuto opportuno renderla pubblica.

Ha un solo metodo astratto: \Methodname{solve}.

Con questo l'intenzione \`e di lasciare libert\`a nella scelta e nell'implementazione dell'algoritmo di soluzione/traversal.

\subsubsection{\Classname{BFSHashmapPuzzle}}
\Classname{BFSHashmapPuzzle} estende \Classname{HashmapPuzzle}.
\Methodname{solve()} \`e implementato attraverso un algoritmo di traversal breadth-first \cite{cormen2001introduction} in cui a ogni iterazione si impostano nel nodo i riferimenti agli oggetti corretti per gli ID.

L'algoritmo \`e stato scelto poich\`e BFS \`e dimostrato essere ottimale in termini di complessit\`a e perch\`e appoggiandosi su una coda appare ingenuamente pi\`u semplice da parallelizzare - come suggerito opzionalmente al punto 5. delle specifiche di progetto\cite{prspec}.
\subsubsection{\Classname{MissingPiecesException}}
\Classname{MissingPiecesException} estende \Classname{Exception} indica che l'algoritmo di risoluzione si blocca perch\`e vi sono dei pezzi mancanti.

\subsubsection{\Classname{PuzzleNotSolvedException}}
\Classname{PuzzleNotSolvedException} estende \Classname{RuntimeException}.
Indica un errore da parte del programmatore nel tentare di accedere o iterare sulla soluzione senza prima avere invocato \Methodname{solve()}, come discusso in \ref{pnse}.
%TODO cit
\subsection{\Pkgname{puzzlesolver.io}}
\subsubsection{\Classname{IPuzzlePrinter}}
\Classname{IPuzzlePrinter} \`e un'interfaccia che definisce un singolo metodo, \Methodname{print(Puzzle p) throws IOException, PuzzleNotSolvedException}, e si applica nell'intenzione a tutte quelle classi che stampano in \emph{qualche modo} un puzzle - su file, in console, in un file grafico (e.g. SVG) 
\subsubsection{\Classname{PlaintextPuzzlePrinter}}
\Classname{PlaintextPuzzlePrinter} implementa \Classname{IPuzzlePrinter} e stampa su file di testo come prescritto nelle specifiche del progetto alla sezione 3 \cite{prspec}.
\subsubsection{\Classname{PuzzleFileParser}}
\Classname{PuzzleFileParser} \`e una classe che fornisce un metodo statico di utilit\`a \Methodname{parseFile} che legge file nel formato prescritto nella sezione 3 delle specifiche di progetto.
Si \`e fatto in modo che il metodo ritorni una lista di semplici struct ``alla C''.
Si \`e preferito infatti mantenerlo completamente disaccoppiato dalla logica del puzzle e dalla sua rappresentazione interna al programma.
Utilizzare una struct ``alla C'' \`e stata una scelta \emph{particolarmente sofferta}, ma le alternative - es. una ``vera'' classe ad hoc o un array con indici numerico sono apparse eccessivamente complesse o scomode per il programmatore, in particolare perch\`e \`e facile confondere l'ordine dei punti cardinali.

\subsubsection{\Classname{MalformedFileException}}
\Classname{MalformedFileException} estende \Classname{Exception} viene sollevata quando il parser incontra un file con sintassi invalida.
\section{Ragioni progettuali}
Come visto nella disanima classe per classe, si \`e cercato di tenere le interfacce (ovvero i contratti, i cosa-fa) come perno centrale della progettazione per ottenere il massimo disaccoppiamento tra componenti e favorire il riuso del codice.
Similmente, encapsulation e information hiding sono garantiti massimizzando l'interazione tramite metodi - specificati nell'interfaccia e dove necessario propri delle classi - ed evitando di esporre dettagli implementativi o attributi delle classi.

\section{Test e collaudo}
Nelle prime fasi di scrittura si sono approntati dei semplici unit test (\texttt{Basic\-Puzzle\-Piece\-Test} e \texttt{BFS\-Hash\-map\-Puzzle\-Test}) per fare da ``parapetto'' contro errori triviali e modifiche involontarie nel comportamento del codice.
Tali unit test non hanno la pretesa di essere particolarmente curati o ben strutturati, ma solo ``buoni abbastanza'' per ottenere lo scopo summenzionato.
Si sono inoltre utilizzati dei file di esempio - alcuni dei quali malformati o altrimenti inaccettabili (presenti nella cartella \texttt{samples}) unitamente al semplice script (\texttt{quicktest.sh}) per effettuare il collaudo finale.

Non \`e stato dato molto peso a errori non triviali nell'input (e.g. grafi di collegamenti non planari e dunque impossibili da ricostruire o collegamenti non corrisposti dal vicino); sono considerati undefined behaviour e evitarli \`e responsabilit\`a dell'utente.

La correttezza del programma non \`e stata dimostrata formalmente - tuttavia, il solo algoritmo non triviale utilizzato finora (BFS) \`e gi\`a dimostrato in letteratura \cite{cormen2001introduction} e si ritiene di poter quindi approcciare il programma con una sufficiente confidenza.

\section{Compilazione e utilizzo}
\begin{verbatim}
$ cd parte1/ 
$ make
$ ./PuzzleSolver input.txt output.txt
\end{verbatim}
oppure
\begin{verbatim}
$ cd parte1/ 
$ make
$ java -jar PuzzleSolver.jar input.txt output.txt
\end{verbatim}

\bibliography{mybib}{}
\bibliographystyle{plain}
\end{document}
