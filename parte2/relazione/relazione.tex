\documentclass[a4paper]{article}
\usepackage[italian]{babel}
\usepackage[T1]{fontenc}
\usepackage[utf8]{inputenc}
\usepackage{cite}
\usepackage{hyperref}
\newcommand{\Classname}[1]{\textsc{#1}}
\newcommand{\Ifacename}[1]{\textsc{#1}}
\newcommand{\Methodname}[1]{\texttt{#1}}
\newcommand{\Pkgname}[1]{\texttt{#1}}
\usepackage{listings} % Include the listings-package
\usepackage{tikz}
\usetikzlibrary{arrows}

\lstset{
language=Java,
basicstyle=\small\sffamily,
numbers=left,
numberstyle=\tiny,
frame=tb,
columns=fullflexible,
showstringspaces=false
}

\usepackage{tikz}
\usetikzlibrary{shapes,arrows}
\tikzstyle{frame} = [
rectangle, draw,
text width=4em, text centered,
minimum height=3em
% minimum width=5em
]
\tikzstyle{line} = [draw, -latex']


\title{PuzzleSolver - Relazione Programmazione Concorrente e Distribuita, Parte 2}
\author{Tobia Tesan - \#1051819}
\begin{document}
\maketitle

\begin{abstract}
La presente relazione dettaglia e motiva le scelte progettuali dell'allegato programma \texttt{PuzzleSolver}, che realizza le specifiche della parte 2 del progetto di Programmazione Concorrente e Distribuita per l'A.A. 2014/2015 \cite{prspec}.
\end{abstract}

\tableofcontents

\section{Vista d'insieme}

Per realizzare le specifiche della parte 2 del progetto sono state necessarie modifiche minimali alla gerarchia delle classi.
\`E stato infatti sufficiente introdurre una classe \classname{ConcurrentHashmapPuzzle} che, come \classname{BFSHashmapPuzzle} estende \classname{HashmapPuzzle}.

L'organizzazione risultante \`e di conseguenza la seguente:

\begin{verbatim}
|-- core
|   |-- BasicPuzzlePiece.java
|   |-- BasicPuzzlePieceTest.java
|   |-- ConcurrentHashmapPuzzle.java
|   |-- ConcurrentHashmapPuzzleTest.java
|   |-- HashmapPuzzle.java
|   |-- IPuzzle.java
|   |-- IPuzzlePiece.java
|   |-- MissingPiecesException.java
|   `-- PuzzleNotSolvedException.java
|-- io
|   |-- IPuzzlePrinter.java
|   |-- MalformedFileException.java
|   |-- PlaintextPuzzlePrinter.java
|   `-- PuzzleFileParser.java
`-- PuzzleSolver.java
\end{verbatim}

\classname{ConcurrentHashmapPuzzle} include una classe annidata \classname{RowLinker} che implementa \Ifacename{Callable}.
\classname{ConcurrentHashmapPuzzle} differisce nell'interfaccia esposta al pubblico nel fatto che prende un argomento di tipo \classname{ExecutorService}, che viene usato per eseguire \classname{RowLinker}.
\`E stato anche necessario fare delle minime modifiche al metodo \methodname{main}, in modo che questo istanzi il medesimo \classname{ExecutorService}.

\section{\classname{ConcurrentHashmapPuzzle}}
\subsection{Algoritmo}
L'idea di base dell'algoritmo scelto consistente nell'iterare parallelamente su ciascuna riga del puzzle e aggiornare i riferimenti da e verso i \emph{soli} vicini S ed E.
\subsection{Implementazione}
L'iterazione per riga \`e implementata nel \interfaceName{Callable} \classname{RowLinker}, di cui \`e riportata di seguito una versione semplificata (in particolare non si gestiscono i pezzi mancanti).

\begin{verbatim}
public Void call() throws MissingPiecesException {

	t = angolo NW

	while (!ultima colonna est) {
		if (!prima riga nord) {
			// Aggiungi riferimento a vicino nord
		}
		if (!ultima riga sud) {
			// Aggiungi riferimento a vicino sud
		}
		if (!prima colonna ovest) {
			// Aggiungi riferimento a vicino ovest
		}

		t = vicino est
	}

	if (!prima riga nord) {
		// Aggiungi riferimento a vicino nord
	}
	if (!ultima riga sud) {
		// Aggiungi riferimento a vicino sud
	}
	if (!prima colonna ovest) {
		// Aggiungi riferimento a vicino ovest
	}

	return null;
}
\end{verbatim}

L'esecuzione parallela ruota intorno a un \classname{ExecutorCompletionService} che istanzia un \classname{RowLinker} per ogni riga e attende, usando il metodo bloccante \ifacename{Future.get()}, che siano completati tutti.
Si \`e fatto riferimento a JCIP §6.3.6.

\begin{verbatim}
private void linkColSE () throws MissingPiecesException {
	assert(NWCorner != null);
	
	IPuzzlePiece it = NWCorner;
	int rows = 0;
	// JCIP
	CompletionService<Void> completionService = new ExecutorCompletionService<Void>(executor);
	while (!it.isSRow()) {
		completionService.submit(new RowLinker(it, pieceHashMap));
		it = pieceHashMap.get(it.getSouthId());
		if (it == null) {
			throw new MissingPiecesException();
		}
		rows++;
	}
	completionService.submit(new RowLinker(it, pieceHashMap));
	rows++;

	try {
		for (int i = 0; i < rows; i++) {
			Future<Void> f = completionService.take();
			f.get();
		}
		// This completes without blocking or raising
		// execptions iff exactly n Callables have completed
	} catch (InterruptedException ie) {
		Thread.currentThread().interrupt();
	} catch (ExecutionException e) {
		Throwable t = e.getCause();
		  if (t instanceof RuntimeException) 
	            throw (RuntimeException) t;
	        else if (t instanceof Error)
	            throw (Error) t;
	        else if (t instanceof MissingPiecesException)
	        	throw (MissingPiecesException) t;
	        else
	            throw new IllegalStateException("Not unchecked", e);
	}
}

\end{verbatim}


\subsection{Thread-safety, interferenze, deadlock}
Non \`e possibile avere interferenze o deadlock all'interno dell'algoritmo di risoluzione.
Infatti, il metodo \classname{Hashmap}.\methodname{get} \`e thread-safe, a patto di non aggiungere elementi alla tavola hash durante la soluzione del puzzle, e ciascun \classname{PuzzlePiece} viene modificato da uno e un solo thread (quello corrispondente alla riga a cui il pezzo appartiene).
Similmente, non \`e possibile ottenere deadlock perch\`e non si presentano le condizioni di Coffman (\`e in particolare immediato che non ci pu\`o essere attesa circolare).

La classe risultante non \`e per\`o (a dispetto del nome) thread-safe, e il suo uso in un programma terzo dovrebbe essere condizionato alla gestione manuale della concorrenza.

Si manifestano in particolare delle race condition sull'attributo \methodname{solved}: sarebbe possibile, ad esempio, avviare \methodname{solve} e aggiungere, frattanto, un pezzo al puzzle con il metodo \methodname{add}: se \methodname{solve} alzasse il flag \methodname{solved} \emph{dopo} la terminazione di \methodname{add}, il puzzle sarebbe marcato come \emph{risolto} trovandosi in realt\`a in uno stato imprevedibile.

\section{Compilazione e utilizzo}
\begin{verbatim}
$ cd parte2/ 
$ make
$ ./puzzlesolver.sh input.txt output.txt
\end{verbatim}
oppure
\begin{verbatim}
$ cd parte2/ 
$ make
$ java -jar PuzzleSolver.jar input.txt output.txt
\end{verbatim}

Opzionalmente:
\begin{verbatim}
$ make javadoc
\end{verbatim}


\bibliography{mybib}{}
\bibliographystyle{plain}
\end{document}
