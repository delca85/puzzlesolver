\documentclass[a4paper]{article}
\usepackage[italian]{babel}
\usepackage[T1]{fontenc}
\usepackage[utf8]{inputenc}
\usepackage{cite}
\usepackage{hyperref}
\newcommand{\Classname}[1]{\textsc{#1}}
\newcommand{\Ifacename}[1]{\textsc{#1}}
\newcommand{\Methodname}[1]{\texttt{#1}}
\newcommand{\Pkgname}[1]{\texttt{#1}}
\usepackage{listings} % Include the listings-package
\usepackage{tikz}
\usetikzlibrary{arrows}

\lstset{
language=Java,
basicstyle=\small\sffamily,
numbers=left,
numberstyle=\tiny,
frame=tb,
columns=fullflexible,
showstringspaces=false
}

\usepackage{tikz}
\usetikzlibrary{shapes,arrows}
\tikzstyle{frame} = [
rectangle, draw,
text width=4em, text centered,
minimum height=3em
% minimum width=5em
]
\tikzstyle{line} = [draw, -latex']


\title{PuzzleSolver - Relazione Programmazione Concorrente e Distribuita, Parte 2}
\author{Tobia Tesan - \#1051819}
\begin{document}
\maketitle

\begin{abstract}
La presente relazione dettaglia e motiva le scelte progettuali dell'allegato programma \texttt{PuzzleSolver}, che realizza le specifiche della parte 2 del progetto di Programmazione Concorrente e Distribuita per l'A.A. 2014/2015.
\end{abstract}

\tableofcontents

\section{Vista d'insieme}

Per realizzare le specifiche della parte 2 del progetto sono state necessarie modifiche minimali alla gerarchia delle classi.
\`E stato infatti sufficiente introdurre una classe \Classname{ConcurrentHashmapPuzzle} che, come \Classname{BFSHashmapPuzzle} estende \Classname{HashmapPuzzle}.

L'organizzazione risultante \`e di conseguenza la seguente:

\begin{verbatim}
|-- core
|   |-- BasicPuzzlePiece.java
|   |-- BasicPuzzlePieceTest.java
|   |-- ConcurrentHashmapPuzzle.java
|   |-- ConcurrentHashmapPuzzleTest.java
|   |-- HashmapPuzzle.java
|   |-- IPuzzle.java
|   |-- IPuzzlePiece.java
|   |-- MissingPiecesException.java
|   `-- PuzzleNotSolvedException.java
|-- io
|   |-- IPuzzlePrinter.java
|   |-- MalformedFileException.java
|   |-- PlaintextPuzzlePrinter.java
|   `-- PuzzleFileParser.java
`-- PuzzleSolver.java
\end{verbatim}

\Classname{ConcurrentHashmapPuzzle} include una classe annidata \Classname{RowLinker} che implementa \Ifacename{Callable}.
\Classname{ConcurrentHashmapPuzzle} differisce nell'interfaccia esposta al pubblico nel fatto che prende un argomento di tipo \Classname{ExecutorService}, che viene usato per eseguire \Classname{RowLinker}.
\`E stato anche necessario fare delle minime modifiche al metodo \Methodname{main}, in modo che questo istanzi il medesimo \Classname{ExecutorService}.

\section{\Classname{ConcurrentHashmapPuzzle}}
\subsection{Algoritmo}
L'idea di base dell'algoritmo scelto consistente nell'iterare parallelamente su ciascuna riga del puzzle e aggiornare i riferimenti verso (ma non \emph{da}) i vicini N, S, W, E.

\subsection{Implementazione}
L'iterazione per riga \`e implementata nel \Ifacename{Callable} \Classname{RowLinker}, di cui \`e riportata in Fig. \ref{rowlinker} una versione semplificata (in particolare non si gestiscono i pezzi mancanti).


\begin{figure}[h!]
  \centering
\begin{lstlisting}[frame=single]
public Void call() throws MissingPiecesException {

	t = angolo NW

	while (!ultima colonna est) {
		if (!prima riga nord) {
                        // Preleva vicino nord da HashMap,
                        // MissingPiecesException se non trovato;

			// Aggiungi riferimento a vicino nord

		}
		if (!ultima riga sud) {
                        // Preleva vicino sud da HashMap,
                        // MissingPiecesException se non trovato;

			// Aggiungi riferimento a vicino sud
		}
		if (!prima colonna ovest) {
                        // ...
		}

		t = vicino est
	}

	if (!prima riga nord) {
                // ...
	}
	if (!ultima riga sud) {
                // ...
	}
	if (!prima colonna ovest) {
                // ...
	}

	return null;
}
\end{lstlisting}
\caption {Scheletro di \textsc{RowLinker.call()}}
\label{rowlinker}
\end{figure}


L'esecuzione parallela ruota intorno a un \Classname{ExecutorCompletionService} che istanzia un \Classname{RowLinker} per ogni riga e attende, usando il metodo bloccante \Ifacename{Future.get()}, che siano completati tutti - in modo sostanzialmente analogo a istanziare manualmente un numero di thread e restare in attesa del loro completamento con una \Methodname{join()}.
Si \`e fatto riferimento a \cite{goetz2006java} (§6.3.6) per l'uso di \Classname{ExecutorCompletionService}.

Il metodo \Classname{Future}.\Methodname{get} pu\`o sollevare \Classname{InterruptedException}, \Classname{ExecutionException}.
\Classname{ExecutionException} \`e un wrapper intorno a eventuali eccezioni sollevate dal \Methodname{call()}, in questo caso \Classname{Linker}.\Methodname{call()}: l'unica eccezione \emph{checked} da esso sollevata \`e \Classname{MissingPiecesException}, che viene estratta e sollevata, insieme alle eccezioni \emph{unchecked}.

Viene inoltre gestita \Classname{InterruptedException} - corrispondente a una interruzione brusca della \Methodname{call()} - interrompendo il thread principale.

\begin{figure}[h!]
  \centering
\label{linkcol}
\begin{lstlisting}[frame=single]
private void linkColSE () throws MissingPiecesException {
	assert(NWCorner != null);
	
	IPuzzlePiece it = NWCorner;
	int rows = 0;
	// JCIP
	CompletionService<Void> completionService = 
             new ExecutorCompletionService<Void>(executor);
	while (!it.isSRow()) {
		completionService.submit(new RowLinker(it, pieceHashMap));
		it = pieceHashMap.get(it.getSouthId());
		if (it == null) {
			throw new MissingPiecesException();
		}
		rows++;
	}
	completionService.submit(new RowLinker(it, pieceHashMap));
	rows++;

	try {
		for (int i = 0; i < rows; i++) {
			Future<Void> f = completionService.take();
			f.get();
		}
		// This completes without blocking or raising
		// execptions iff exactly n Callables have completed
	} catch (InterruptedException ie) {
		Thread.currentThread().interrupt();
	} catch (ExecutionException e) {
		Throwable t = e.getCause();
		// Unchecked
		if (t instanceof RuntimeException) 
			throw (RuntimeException) t;
		else if (t instanceof Error)
			throw (Error) t;
		// Checked
		else if (t instanceof MissingPiecesException)
			throw (MissingPiecesException) t;
	}
}
\end{lstlisting}
\caption {\textsc{ConcurrentHashmapPuzzle.linkColSE()}}
\end{figure}


\subsection{Uniformit\`a}
Poich\`e i puzzle sono rettangolari, ciascuna istanza di \Classname{RowLinker} iterer\`a sullo stesso numero di pezzi.

\`E dunque realizzato il requisito di uniformit\`a per puzzle di dimensioni $n \times m$ con $n, m \gg 1$.

Sono tuttavia casi degeneri quelli in cui il puzzle \`e formato da una sola riga o una sola colonna.
Nel caso di una sola riga, l'algoritmo non presenterebbe, di fatto, parallelismo.
Nel caso di una sola colonna di dimensione $n$, si avrebbero $n$ thread che opererebbero su un solo pezzo, causando un grande overhead.

\subsection{Thread-safety, interferenze, deadlock}
\emph{Non \`e possibile avere interferenze o deadlock all'interno dell'algoritmo di risoluzione}.

Infatti, il metodo \Classname{Hashmap}.\Methodname{get()} \`e thread-safe, a patto di non aggiungere elementi alla tavola hash durante la soluzione del puzzle, e ciascun \Classname{PuzzlePiece} viene modificato da uno e un solo thread (quello corrispondente alla riga a cui il pezzo appartiene).

Similmente, non \`e possibile ottenere deadlock perch\`e non si presentano le condizioni di Coffman (\`e in particolare immediato che non ci pu\`o essere attesa circolare) \cite{tanenbaum2001modern}.

\paragraph{Thread-safety} La classe risultante non \`e per\`o (a dispetto del nome) thread-safe, e il suo uso in un programma terzo dovrebbe essere condizionato alla gestione manuale della concorrenza.

Si manifestano in particolare delle race condition sull'attributo \Methodname{solved}: sarebbe possibile, ad esempio, avviare \Methodname{solve} e aggiungere, frattanto, un pezzo al puzzle con il metodo \Methodname{add}: se \Methodname{solve} alzasse il flag \Methodname{solved} \emph{dopo} la terminazione di \Methodname{add}, il puzzle sarebbe marcato come \emph{risolto} trovandosi in realt\`a in uno stato imprevedibile.



\section{Compilazione e utilizzo}
\begin{verbatim}
$ cd parte2/ 
$ make
$ ./puzzlesolver.sh input.txt output.txt
\end{verbatim}
oppure
\begin{verbatim}
$ cd parte2/ 
$ make
$ java -jar PuzzleSolver.jar input.txt output.txt
\end{verbatim}

Opzionalmente:
\begin{verbatim}
$ make javadoc
\end{verbatim}


\bibliography{mybib}{}
\bibliographystyle{alpha}
\end{document}
