\documentclass[a4paper]{article}

\usepackage[italian]{babel}
\usepackage[T1]{fontenc}
\usepackage[utf8]{inputenc}
\usepackage{cite}
\usepackage{hyperref}
\newcommand{\Classname}[1]{\textsc{#1}}
\newcommand{\Ifacename}[1]{\textsc{#1}}
\newcommand{\Methodname}[1]{\texttt{#1}}
\newcommand{\Pkgname}[1]{\texttt{#1}}
\usepackage{listings} % Include the listings-package
\usepackage{tikz}
\usetikzlibrary{arrows}

\lstset{
language=Java,
basicstyle=\small\sffamily,
numbers=left,
numberstyle=\tiny,
frame=tb,
columns=fullflexible,
showstringspaces=false
}

\usepackage{tikz}
\usetikzlibrary{shapes,arrows}
\tikzstyle{frame} = [
rectangle, draw,
text width=4em, text centered,
minimum height=3em
% minimum width=5em
]
\tikzstyle{line} = [draw, -latex']


\title{PuzzleSolver - Relazione Programmazione Concorrente e Distribuita, Parte 3}
\author{Tobia Tesan - \#1051819}
\begin{document}
\maketitle

\begin{abstract}
La presente relazione dettaglia e motiva le scelte progettuali dei programmi \texttt{PuzzleSolverServer} e \texttt{PuzzleSolverClient}, che realizzano le specifiche della parte 3 del progetto di Programmazione Concorrente e Distribuita per l'A.A. 2014/2015.
\end{abstract}

\tableofcontents

\section{Vista d'insieme}
\subsection{Premessa}
Implementare una strategia di comunicazione client/server efficiente senza modifiche drastiche alla struttura delle classi non e' stato del tutto triviale a causa delle peculiarit\`a delle classi e strutture dati impiegate nelle parti 1 e 2.

In particolare, nella parte 1 si \`e concepita un interfaccia \Ifacename{PuzzlePrinter} con un metodo \Methodname{print} che prendeva come argomento un \Ifacename{IPuzzle} risolto, la cui implementazione stampava in console la soluzione del puzzle.
Per utilizzarla occorre dunque una istanza di \Ifacename{IPuzzle} sul client.

D'altra parte l'implementazione di \Ifacename{IPuzzle} prevista in parte 1 e parte 2 si presta molto male ad essere serializzata e trasferita al client una volta risolta, poich\`e contiene una \Classname{HashMap} di \Classname{BasicPuzzlePiece}, ognuno dei quali contiene quattro riferimenti ai vicini che si perdono nella serializzazione.
\label{intro}
Si \`e d'altronde preferito evitare di agire ridefinendo \Methodname{readObject} e \Methodname{writeObject} per correggere la situazione e si \`e invece agito come dettagliato in seguito.

\subsection{Strategia di comunicazione}
La strategia di comunicazione scelta \`e la seguente:
\begin{enumerate}
\item Il programma server istanzia un oggetto \Ifacename{IRemotePuzzle}.
\item Il programma client recupera un riferimento remoto all'oggetto \Ifacename{IRemotePuzzle}
\item Il programma client legge il file di input con \Classname{PuzzleFileParser}, immutato dalla prima versione, costruisce localmente un \Ifacename{IPuzzlePiece} e chiama il metodo \Methodname{addPiece} sull'oggetto remoto.
\item Il programma client chiama \Methodname{solve()} sull'oggetto remoto
\item \label{foo} Il programma client ottiene un oggetto locale \Ifacename{IPuzzle} attraverso il metodo \Ifacename{IRemotePuzzle}.\Methodname{freeze()}.
\item Il programma client stampa il risultato attraverso la classe \Classname{PlaintextPuzzlePrinter}, immutata dalla prima versione, passando l'oggetto locale recuperato al punto precedente.
\end{enumerate}

Il passo \ref{foo} \`e reso necessario dal fatto che, come detto in \ref{intro}, l'oggetto remoto non \`e serializzabile.
Il metodo \Methodname{freeze} restituisce un oggetto serializzabile che pu\`o essere stampato localmente.

\subsection{Robustezza}
Il programma client avvolge il riferimento remoto \Ifacename{IRemotePuzzle} con una classe \Classname{ExponentialBackoffPuzzleWrapper}, che prova a fare backoff esponenziale sulle chiamate ai metodi dell'oggetto remoto prima di lasciare risalire la \Classname{RemoteException}.

\section{Organizzazione delle classi}

\`E riportato di seguito l'elenco delle classi.
Le nuove classi sono marcate da \texttt{*}

\begin{verbatim}
puzzlesolver
|-- client
|   `-- ExponentialBackoffPuzzleWrapper.java *
|-- core
|   |-- ArrayPuzzle.java *
|   |-- BasicPuzzlePiece.java
|   |-- BasicPuzzlePieceTest.java
|   |-- ConcurrentHashmapPuzzle.java
|   |-- ConcurrentHashmapPuzzleTest.java
|   |-- HashmapPuzzle.java
|   |-- IPuzzle.java
|   |-- IPuzzlePiece.java
|   |-- MissingPiecesException.java
|   `-- PuzzleNotSolvedException.java
|-- io
|   |-- IPuzzlePrinter.java
|   |-- MalformedFileException.java
|   |-- PlaintextPuzzlePrinter.java
|   `-- PuzzleFileParser.java
|-- PuzzleSolverClient.java *
|-- PuzzleSolverServer.java *
`-- server
    |-- FreezableHashmapPuzzle.java *
    |-- FrozenArrayPuzzle.java *
    |-- IRemotePuzzle.java *
    `-- RemoteHashmapPuzzle.java *
\end{verbatim}

\subsection{Nuove classi}
\subsubsection{puzzlesolver.server.IRemotePuzzle}
L'interfaccia \Ifacename{IRemotePuzzle} offre gli stessi metodi di \Ifacename{IPuzzle} con l'aggiunta della clausola \texttt{throws } \Ifacename{RemoteException} ed \`e dunque idonea ad essere utilizzata con RMI.
\subsubsection{puzzlesolver.client.ExponentialBackoffPuzzleWrapper}
\Classname{ExponentialBackoffPuzzleWrapper} \`e un wrapper intorno a un \Ifacename{RemotePuzzle} che prova a fare backoff esponenziale un numero \texttt{MAX_RETRIES} di volte prima di rilanciare l'eccezione \Classname{RemoteException}.
\subsubsection{puzzlesolver.server.FreezableHashmapPuzzle}
\Classname{FreezableHashmapPuzzle} estende \Classname{ConcurrentHashmapPuzzle} aggiungendo un metodo \Methodname{freeze} che restituisce un \emph{nuovo} \Ifacename{IPuzzle} risolto e ``congelato''.
\subsubsection{puzzlesolver.server.RemoteHashmapPuzzle}
\Classname{RemoteHashmapPuzzle} estende \Classname{UnicastRemoteObject} e implementa \Classname{IRemotePuzzle}.
\`E sostanzialmente un wrapper intorno a un \Classname{FreezableHashmapPuzzle}.
\subsubsection{puzzlesolver.core.ArrayPuzzle}
\Classname{ArrayPuzzle} \`e una classe \emph{astratta} che rappresenta un puzzle sotto forma di array.
\subsubsection{puzzlesolver.server.FrozenArrayPuzzle}
\Classname{FrozenArrayPuzzle} \`e un'estensione minimale di \Classname{ArrayPuzzle}.
Solleva una \Classname{RuntimeException} se si tenta di effettuare operazioni in scrittura.

\subsection{Modifiche a classi esistenti}
\`E stato possibile evitare di fare modifiche a classi esistenti, con l'eccezione di avere esplicitamente dichiarato \Ifacename{Serializable} la classe \Classname{BasicPuzzlePiece}.

\section{Compilazione e utilizzo}
\begin{verbatim}
$ cd parte3/ 
$ make
$ ./puzzlesolverserver.sh localhost &
$ ./puzzlesolverclient.sh input.txt output.txt localhost
\end{verbatim}
oppure
\begin{verbatim}
$ cd parte2/ 
$ make
$ java -jar PuzzleSolverServer.jar localhost &
$ java -jar PuzzleSolverClient.jar input.txt output.txt localhost
\end{verbatim}

% \bibliography{mybib}{}
% \bibliographystyle{alpha}
\appendix
\section{Addendum alla parte 2}
Nella parte 2 si \`e scelto di cambiare l'algoritmo risolutivo poich\`e l'attraversamento \emph{breadth-first} si adatta molto male ad essere parallelizzato.
Nello specifico, il fatto che la struttura dati scelta \emph{non sia} un albero ma lo contenga solamente come sottoinsieme dei suoi spigoli e vertici, presentando dunque cicli e diversi percorsi tra due nodi, fa s\`i che sia molto difficile partizionare il lavoro in unit\`a indipendenti e fare in modo che ciascun thread sappia quando fermarsi.
\end{document}
